\documentclass[10pt]{article}
 
\usepackage[margin=1in]{geometry} 
\usepackage{amsmath,amsthm,amssymb}
 
\newcommand{\N}{\mathbb{N}}
\newcommand{\Z}{\mathbb{Z}}
 
\newenvironment{theorem}[2][Theorem]{\begin{trivlist}
\item[\hskip \labelsep {\bfseries #1}\hskip \labelsep {\bfseries #2.}]}{\end{trivlist}}
\newenvironment{lemma}[2][Lemma]{\begin{trivlist}
\item[\hskip \labelsep {\bfseries #1}\hskip \labelsep {\bfseries #2.}]}{\end{trivlist}}
\newenvironment{exercise}[2][Exercise]{\begin{trivlist}
\item[\hskip \labelsep {\bfseries #1}\hskip \labelsep {\bfseries #2.}]}{\end{trivlist}}
\newenvironment{problem}[2][Problem]{\begin{trivlist}
\item[\hskip \labelsep {\bfseries #1}\hskip \labelsep {\bfseries #2.}]}{\end{trivlist}}
\newenvironment{question}[2][Question]{\begin{trivlist}
\item[\hskip \labelsep {\bfseries #1}\hskip \labelsep {\bfseries #2.}]}{\end{trivlist}}
\newenvironment{corollary}[2][Corollary]{\begin{trivlist}
\item[\hskip \labelsep {\bfseries #1}\hskip \labelsep {\bfseries #2.}]}{\end{trivlist}}

\newenvironment{solution}{\begin{proof}[Solution]}{\end{proof}}
 
\begin{document}
 
\title{ASTR 400B Set 3}
\author{William Lake
}

\maketitle

\section*{Table}

\begin{center}
\centering
\begin{tabular}{||c c c c c c||} 
 \hline
 Galaxy Name & Halo Mass (\(10^{12} M_\odot\)) & Disk Mass (\(10^{12} M_\odot\)) & Bulge Mass (\(10^{12} M_\odot\)) & Total Mass (\(10^{12} M_\odot\)) & $f_{\text{bar}}$ \\ [0.5ex] 
 \hline\hline
 Milky Way & \(1.975\) & \(7.5 * 10^{-2}\) & \(1.001 * 10^{-2}\) & 2.06 & \(0.024\) \\ 
 \hline
 M31 & \(1.921\) & \(1.2 * 10^{-1}\) & \(1.905 * 10^{-2}\) & 2.06 & \(0.067\)\\
 \hline
 M33 & \(1.866 * 10^{-1}\) & \(9.3 * 10^{-3}\) & 0 & \(1.959 * 10^{-1}\) & \(0.047\)\\ 
 \hline
 Local Group & \(4.083\) & \(2.043 * 10^{-1}\) & \(2.906 * 10^{-2}\) & \(4.316\) & \(0.054\)\\ 
 \hline
\end{tabular}
\end{center}

\section*{Questions}

\begin{enumerate}
\item The total masses of M31 and the Milky Way are the same in this simulation. They are dominated by the masses of their Halos.
\item The stellar mass of M31 is somewhat larger than that of the Milky Way, which implies that M31 will be more luminous. 
\item The Milky Way has 2.8\% more dark matter than M31. This is mildly surprising, given that M31 has more stellar matter, which one would imagine would indicate a higher concentration of dark matter, from gravitational considerations.
\item The Milky Way has a baryon fraction of 0.024, M31 0.067, and M33 0.047. These are significantly lower than the 15\% observed universally. This could be because significant portions of stellar mass (but not of dark matter) are present in stellar clusters, as opposed to galaxies. 

\end{enumerate}

\end{document}
